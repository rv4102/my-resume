%------------------------
% Resume Template
% Author : Anubhav Singh
% Github : https://github.com/xprilion
% License : MIT
%------------------------

% Obtained from Subhajyoti Halder

\documentclass[a4paper,20pt]{article}

\usepackage{latexsym}
\usepackage[empty]{fullpage}
\usepackage{titlesec}
\usepackage{marvosym}
\usepackage[usenames,dvipsnames]{color}
\usepackage{verbatim}
\usepackage{enumitem}
\usepackage[pdftex,pdfpagelabels,bookmarks,hyperindex,hyperfigures]{hyperref}
\usepackage{fancyhdr}
\usepackage{multicol}
\usepackage{fontawesome5}

\hypersetup{
    colorlinks=true,
    linkcolor=black,
    filecolor=black,      
    urlcolor=black,
    }

\pagestyle{fancy}
\fancyhf{} % clear all header and footer fields
\fancyfoot{}
\renewcommand{\headrulewidth}{0pt}
\renewcommand{\footrulewidth}{0pt}

% Adjust margins
\addtolength{\oddsidemargin}{-0.530in}
\addtolength{\evensidemargin}{-0.375in}
\addtolength{\textwidth}{1in}
\addtolength{\topmargin}{-.45in}
\addtolength{\textheight}{1in}

\urlstyle{rm}

\raggedbottom
\raggedright
\setlength{\tabcolsep}{0in}

% Sections formatting
\titleformat{\section}{
  \vspace{-10pt}\scshape\raggedright\large
}{}{0em}{}[\color{black}\titlerule \vspace{-6pt}]

%-------------------------
% Custom commands
\newcommand{\resumeItem}[2]{
  \item\small{
    \textbf{#1}{#2 \vspace{-2pt}}
  }
}

\newcommand{\resumeItemWithoutTitle}[1]{
  \item\small{
    {\vspace{-2pt}}
  }
}

\newcommand{\resumeSubheading}[4]{
  \vspace{-1pt}\item
    \begin{tabular*}{0.97\textwidth}{l@{\extracolsep{\fill}}r}
      \textbf{#1} & #2 \\
      \textit{#3} & \textit{#4} \\
    \end{tabular*}\vspace{-5pt}
}

\newcommand{\honorsSubheading}[2]{
  \vspace{-1pt}\item
    \begin{tabular*}{0.97\textwidth}{l@{\extracolsep{\fill}}r}
      \textbf{#1} & #2 \\
    \end{tabular*}\vspace{-15pt}
}


\newcommand{\resumeSubItem}[2]{\resumeItem{#1}{#2}\vspace{-5pt}}

\renewcommand{\labelitemii}{$\circ$}

\newcommand{\resumeSubHeadingListStart}{\begin{itemize}[leftmargin=*]}
\newcommand{\resumeSubHeadingListEnd}{\end{itemize}}
\newcommand{\resumeItemListStart}{\begin{itemize}}
\newcommand{\resumeItemListEnd}{\end{itemize}\vspace{-7pt}}

%-----------------------------
%%%%%%  CV STARTS HERE  %%%%%%

\begin{document}


%----------HEADING-----------------
\begin{tabular*}{\textwidth}{l@{\extracolsep{\fill}}r}
  \textbf{{\LARGE Rushil Venkateswar}} & Email:~~\href{mailto:rushilv14@gmail.com}{rushilv14@gmail.com}\\
  & Mobile:~~+91-8789-309-659 \\
  & Github:~~\url{https://github.com/rv4102/} \\
\end{tabular*}

%-----------SUMMARY-----------------
% \section{~~Summary}
%     \resumeSubHeadingListStart
%     \resumeSubheading
%       {\justifying Pursuing B.Tech (3rd Year) from Dept. of Computer Science \& Engineering at IIT-Kharagpur and keen on working in Projects and Interning at industrial workspaces. He has worked on NLP-based Event trigger detection, computer vision task like object-detection on Image. Subhajyoti is a KVPY scholar and NTSE fellowship awardee. He is a rank-holder in numerous National and State-level competitive exams (IIT-JEE, JEE-Mains, WBJEE etc.). Apart from academics, he is also the Hall Secretary: Sports and Games and an active member at spreading education and social hygiene awareness in rural areas under NSS IIT-Kgp.}{}
%       {}{}
%     \resumeSubHeadingListEnd

%-----------EDUCATION-----------------
\section{Education}
  \resumeSubHeadingListStart
    \resumeSubheading
    {Indian Institute of Technology Kharagpur}{Kharagpur, India}
    {Dual Degree (B. Tech. + M. Tech.) - Computer Science and Engineering;  CGPA: 8.79}{Dec 2020 - ongoing}
     
    \resumeSubheading
    {Little Flower School}{Jamshedpur, India}
    {Higher Secondary - ISC;  percentage: 96.25\%}{Mar 2018 - Mar 2020}
    {\scriptsize \textit{ \footnotesize{\newline{}\textbf{Subjects:} Mathematics, Physics, Chemistry, Computer Sc. (Java), English }}}
     
    \resumeSubheading
    {Little Flower School}{Jamshedpur, India}
    {Secondary - ICSE;  percentage: 95.40\%}{Mar 2007 - Mar 2018}
  \resumeSubHeadingListEnd


%-----------PROJECTS-----------------
\vspace{0pt}
\section{Projects}
  \resumeSubHeadingListStart
    \resumeSubheading{Students' Auditorium Management System (\textit{CS29202: Software Engineering})}{\href{https://github.com/rv4102/AudiBooking}{\faGithub}}
    {Front-end: HTML, CSS, Javascript \hspace{4mm} Backend: Flask \hspace{4mm} Database: SQLite}{Spring 2022}
    \resumeItemListStart
      \resumeItem\textmd{An \textbf{auditorium management website} which runs on localhost (127.0.0.1) and allows the user to maintain a database of shows to be screened, along with providing functionality for seat booking.}
      \resumeItem\textmd{\textbf{OTP-based login} for users of the software is supported.}
    \resumeItemListEnd

   
    \resumeSubheading{KGP-miniRISC Processor (\textit{CS39001: Computer Organisation \& Architecture})}{\href{https://github.com/rv4102/CompOrg-Lab/tree/main/A6_KGP_miniRISC_Processor}{\faGithub}}
    {Tech: Verilog (HDL), AMD Xilinx ISE, Nexys Artix-7 FPGA}{Autumn 2022}
    \resumeItemListStart
      \resumeItem\textmd{Developed a 32-bit word-length \textbf{single-cycle} instruction execution unit with a purely combinational design.}
      \resumeItem\textmd{Successfully dumped the bitstream onto a \textbf{Nexys A7 FPGA board} using Xilinx ISE and ran programs (written in assembly) such as sorting and linear search.}
      % \resumeItem\textmd{Clock frequency determined by the single longest instruction in the given ISA.}
    \resumeItemListEnd

   
    \resumeSubheading{tinyC Compiler \textit{(CS39003: Compilers)}}{\href{https://github.com/rv4102/Compilers-Lab}{\faGithub}}
    {Tech: Flex, Bison}{Autumn 2022}
    \resumeItemListStart
      \resumeItem\textmd{Created library for standard input-output operations and followed International Standard ISO/IEC 9899:1999 (E).}
      \resumeItem\textmd{Defined flex specifications for the language of tinyC using the Phase Structure Grammar given in the C Standard.}
      \resumeItem\textmd{Used Bison specifications to define the tokens of tinyC and write the semantic actions in Bison to translate a tinyC program into an array of 3-address quad’s, a supporting symbol table, and other auxiliary data structures.}
      \resumeItem\textmd{Developed target code translator to generate the assembly language of x86-64  from the Three-Address-Code quad array.}
    \resumeItemListEnd
   
    \resumeSubheading{Instance Segmentation and Detection (\textit{CS29202: Software Engineering})}{\href{https://github.com/rv4102/CS29202-Codes/tree/main/A4}{\faGithub}}
    {Tech: MaskRCNN (Pytorch), PIL, Tkinter}{Spring 2022}
    \resumeItemListStart
      \resumeItem\textmd{Developed a \textbf{Tkinter based GUI} to display bounding boxes or segmentation masks for the image selected.}
      \resumeItem\textmd{Used a pre-trained MaskRCNN model from Pytorch library to generate \textbf{masks} and \textbf{bounding boxes} for a given image and used \textbf{matplotlib} to plot them.}
      \resumeItem\textmd{Created a \textbf{python package} from source code.}
    \resumeItemListEnd
  \resumeSubHeadingListEnd


%-----------EXPERIENCE---------------
\vspace{0pt}
\section{Experience}
  \resumeSubHeadingListStart
    \resumeSubheading{Maternal \& Child Health Monitoring | Stanford University | Prof. Pascal Geldsetzer}{Kharagpur, India}
    {Objective: Estimate key indicators of health status in low income countries using satellite imagery}{May '23 - Aug '23}
    \resumeItemListStart
      \resumeItem\textmd{Trained a \textbf{random forest regressor} for \textbf{multi-output regression} using \textbf{11000 numerical features} and performed \textbf{K-Fold Cross Validation}}
      \resumeItem\textmd{Used Protégé for making an ontology for Python language. Explored OWLReady python package to create ontologies.}
      \resumeItem\textmd{Implemented Frequent Itemset Mining and used it to perform query expansion.}
    \resumeItemListEnd

    \resumeSubheading{Research Intern | Prof. Partha Pratim Das, IIT Kharagpur}{Kharagpur, India}
    {Topic: Development of a Python Tutor}{May 2022 - July 2022}
    \resumeItemListStart
      \resumeItem\textmd{Tech: Python}
      \resumeItem\textmd{Used Protégé for making an ontology for Python language. Explored OWLReady python package to create ontologies.}
      \resumeItem\textmd{Implemented Frequent Itemset Mining and used it to perform query expansion.}
    \resumeItemListEnd
    
    \resumeSubheading{Research Intern | Dr. Debasish Chakraborty, ISRO}{Kharagpur, India}
    {Topic: Semantic Segmentation of Remote Sensing Images \href{https://github.com/rv4102/qnet.git}{\faGithub}}{Apr 2022 - Dec 2022}
    \resumeItemListStart
      \resumeItem\textmd{Tech: Python, TensorFlow, Keras, OpenCV}
      \resumeItem\textmd{Studied existing literature on CNNs, deep learning and their usage in the remote sensing context.}
      \resumeItem\textmd{Using the Tensorflow framework, built a U-Net like model with improved performance and a fraction of the number of parameters as the original U-Net.}
      \resumeItem\textmd{Created various other scripts to perform inference on variable sized images. Tested on Indian context images taken from Google Earth.}
    \resumeItemListEnd
  \resumeSubHeadingListEnd
\vspace{-5pt}

\section{Coursework}
\begin{description}[font=$\bullet$]
  \item {\textbf{Computer Science (Theory+Lab):} Algorithms-I, Software Engineering, Systems Programming, Compilers, Computer Architecture and Organization, Machine Learning, Algorithms-II, Programming \& Data Structures}
  \vspace{-3pt}
  \item {\textbf{Mathematics:} Discrete Structures, Advanced Calculus, Linear Algebra, Numerical and Complex Analysis, Probability and Statistics, Econometric Analysis, Statistical Inference}
  \vspace{-3pt}
  \item {\textbf{MOOCs:} Machine Learning \textbf{(Andrew Ng)}, Deep Learning Specialization \textbf{(DeepLearning.AI)}}
  % \vspace{-3pt}
  % \item {\textbf{Ongoing Courses:} Deep Learning, Database Management and Systems, Operating Systems, Computer Networks}
  % \vspace{-3pt}
  % \item {\textbf{Finance and Statistics:} Probability and Statistics, Econometric Analysis, Rural Infra. Dev. and Management.}
\end{description}


%\section{Publications}
%\resumeSubHeadingListStart
%\resumeSubItem{Book: Deep Learning on Web (Web Development, Deep Learning)}{Work in Progress book to be published by Packt Publishing in late 2019. Tech: Django, Python, AWS, GCP, Azure (November '18)}
%\vspace{2pt}
%\resumeSubItem{Book: Deep Learning on Mobile Devices (Flutter App Development, Deep Learning)}{Work in Progress book to be published by Packt Publishing in late 2019. Tech: Flutter, Android, Firebase, TensorFlow, Python, Dart (December '18)}
%\resumeSubHeadingListEnd

\section{Skills Summary}
\resumeSubHeadingListStart
  \resumeSubItem{Languages: }{C, C++, Python, MySQL, LaTeX, MATLAB, MIPS Verilog, Bash.}
  \resumeSubItem{Libraries and Frameworks:}{~STL(C++), NumPy, Pandas, Tkinter, PIL, Pytorch, TensorFlow.}
  \resumeSubItem{Platforms \& Tools:}{~Linux, macOS, Git, Jupyter Notebook.}
\resumeSubHeadingListEnd

\vspace{0pt}
\vspace{0pt}

%-----------Awards-----------------
\section{Honors and Awards}
\resumeSubHeadingListStart
  \honorsSubheading{\textbf{All India Rank 473} among 160k candidates}{\emph{JEE Advanced}, Sept 2020}
  % \vspace{-3pt}
  \honorsSubheading{\textbf{All India Rank 1176} among 1.12 million candidates}{\emph{JEE Mains}, Jan 2020}
  % \vspace{-3pt}
  \honorsSubheading {\textbf{KVPY} round 1 qualified}{\emph{IISER Kolkata}, Feb 2019}
  % \vspace{-3pt}
  \honorsSubheading {\textbf{Inter IIT Tech Meet 11.0} solo gold medalist}{\emph{IIT Kanpur}, Feb 2023}
  % \vspace{-3pt}
  \honorsSubheading {\textbf{Optiver Winter School} participant}{\emph{IIT Delhi}, Jan 2023} % one of 300 selected from 3000 IITians
  % \vspace{-3pt}
  \honorsSubheading {\textbf{ACM ICPC Preliminary Round} rank 498}{\emph{IIT Kharagpur}, Nov 2022}
  % \vspace{-3pt}
  % \item {\textbf{National Science Day Quiz} First position | \emph{Gymkhana, IIT Kharagpur} - Feb 2021}
  % \vspace{-3pt}
  % \item {\textbf{Science of Paper Folding Quiz} Second position | \emph{Space Technology Students' Society, IIT Kharagpur} - Aug 2022}

\resumeSubHeadingListEnd

\vspace{5pt}
\vspace{5pt}

\section{Position of Responsibility}
  \resumeSubHeadingListStart
    \resumeSubheading
    {Associate Member}{}
    {Business Club, Indian Institute of Technology Kharagpur}{Dec 2020 - Sept 2021}
    \resumeItemListStart
      \resumeItem\textmd{Authored a \textbf{Whitepaper on XGBoost} {\href{https://drive.google.com/file/d/1e6BMMkBewwBbj1vg58pLpTyw0jNMvRba/view?usp=sharing}{\faDropbox}}.}
      \resumeItem\textmd{Conducted a webinar on Cryptocurrency with Dr. Akshat Shrivastava as the guest lecturer.}
      \resumeItem\textmd{Participated in an intra-club case study competition and guided newly inducted teammates.}
    \resumeItemListEnd
  \resumeSubHeadingListEnd
\vspace{5pt}

\vspace{0pt}
\section{Volunteer Experience}
  \resumeSubHeadingListStart
    \resumeSubheading
    {SWG Mentor (Mentor-Mentee Programme)}{}
    {Students' Welfare Group, Indian Institute of Technology Kharagpur}{}
% \resumeSubheading
%     {Active member under National Service Scheme (NSS Unit-5) IIT-Kgp}{Kharagpur, India}
%     {Actively participated and co-ordinated in translating transcripts into native languages.}{Dec 2020 - May 2022}
  \resumeSubHeadingListEnd
\vspace{5pt}


\end{document}